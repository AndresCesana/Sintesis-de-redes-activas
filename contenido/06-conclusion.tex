\newpage
\section{Conclusión}
En el presente trabajo práctico se abordó el diseño y simulación de circuitos osciladores y filtros activos. Respecto a la generación de señales, se verificó el funcionamiento de un oscilador de anillo de 13 etapas, logrando una frecuencia de oscilación estable de 1.62 MHz, validando la relación inversa entre el número de etapas y la frecuencia de salida.

En cuanto al filtrado analógico, se diseñó un filtro pasabanda tipo Butterworth. Si bien las simulaciones en Python validaron la respuesta en frecuencia teórica centrada en 1 kHz, se observaron discrepancias en la implementación circuital en LTSpice. El circuito propuesto presentó una atenuación de inserción de aproximadamente 6 dB. Esto evidencia la importancia de verificar que la cantidad de etapas físicas (amplificadores operacionales) coincida con el orden del filtro calculado por software para garantizar la selectividad y ganancia deseadas en la banda de paso.