\section{Diseño de filtro activo}
Para este ejemplo se compararán las respuestas de simulación y código en el diseño de un filtro pasabanda activo.
\subsection{Diseño de filtro activo mediante Filter Design}
El diseño del filtro se presentará mediante la herramienta filter design.
\begin{figure}[H]
    \centering
    \includegraphics[width=0.85\linewidth]{Imagenes/filter design.png}
    \caption{Diseño del filtro mediante filter design}
    \label{fig:placeholder}
\end{figure}
Se puede ver que cumple con los requerimientos propuestos.
\subsection{Diseño de filtro activo mediante código en Python}
Mediante el código de python se puede ver la similitud con el realizado en filter design. En este caso se encontró el valor de la ecuación de Chevychev. 
\[
H(s)=
\frac{7.99\times10^{6}\, s^2}
{s^4
+ 957.2\, s^3
+ 4.75\times10^{7}\, s^2
+ 3.78\times10^{10}\, s
+ 1.56\times10^{15}}
\]
Y la respuesta en frecuencia es:
\begin{figure}[H]
    \centering
    \includegraphics[width=0.65\linewidth]{Imagenes/Diseño en python.png}
    \caption{Diseño del filtro mediante Python}
    \label{fig:placeholder}
\end{figure}
\subsection{Circuito del filtro activo}
El circuito propuesto para cumplir con las especificaciones es el siguiente:
\begin{figure}[H]
    \centering
    \includegraphics[width=0.75\linewidth]{Imagenes/circuito.png}
    \caption{Circuito del filtro activo}
    \label{fig:placeholder}
\end{figure}
\subsection{Respuesta en LTSpice para el circuito propuesto}
Como respuesta del filtro elegido tenemos el siguiente gráfico.
\begin{figure}[H]
    \centering
    \includegraphics[width=0.5\linewidth]{Imagenes/lt.png}
    \caption{Respuesta del circuito simulado en LTSpice}
    \label{fig:placeholder}
\end{figure}

Se puede encontrar una gran similitud de las respuestas tanto teórica, como calculada y simulada.