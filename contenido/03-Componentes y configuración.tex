\subsection{Componentes y Configuración}

\subsection*{Diseño del Oscilador de Anillo}

Para este trabajo práctico, se diseñará un oscilador de anillo utilizando 13 inversores conectados en serie. El diseño y simulación se realizarán en el software LTspice, considerando las siguientes especificaciones:

\begin{itemize}
    \item \textbf{Número de etapas:} $n = 13$ (impar para garantizar la oscilación).
    \item \textbf{Tensión de alimentación:} $V_{DD} = 1\text{V}$.
    \item \textbf{Tiempo de propagación por inversor:} $t_p = 10\text{ns}$.
    \item \textbf{Nivel de disparo ($V_M$):} $V_M = 0.5 \cdot V_{DD} = 0.5\text{V}$.
    \item \textbf{Capacidad de carga por inversor:} $C_L = 10\text{pF}$.
    \item \textbf{Criterio del 50\%:} Se tomará como referencia el cruce por el nivel $V_M$ para calcular el retardo y evaluar la oscilación.
\end{itemize}

\subsection*{Cálculo de la Frecuencia de Oscilación}

La frecuencia de oscilación ($f$) de un oscilador de anillo se determina a partir del tiempo de propagación ($t_p$) de cada etapa y el número total de etapas ($n$):

\[
f = \frac{1}{2n \cdot t_p}
\]

Sustituyendo los valores especificados:

\[
f = \frac{1}{2 \cdot 13 \cdot 10 \text{ ns}} = \frac{1}{260 \text{ ns}} \approx 3.85\text{ MHz}
\]
