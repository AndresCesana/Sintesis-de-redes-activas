\documentclass[12pt,a4paper]{article}

% ---------------- Idioma y tipografía ----------------
\usepackage[spanish]{babel}
\usepackage[utf8]{inputenc}
\usepackage[T1]{fontenc}
\usepackage{lmodern}

% ---------------- Formato ----------------
\usepackage{geometry}
\geometry{margin=2.5cm}
\usepackage{setspace}
\onehalfspacing

% ---------------- Matemática ----------------
\usepackage{amsmath,amssymb}

% ---------------- Figuras ----------------
\usepackage{graphicx}
\usepackage{float}

% ---------------- Tablas ----------------
\usepackage{booktabs}
\usepackage{array}

% ---------------- Hipervínculos ----------------
\usepackage{hyperref}

\begin{document}

\begin{titlepage}
\centering
{\LARGE Síntesis De Redes Activas\par}
\vspace{0.6cm}
{\Large Trabajo Práctico de Laboratorio\par}
\vspace{0.2cm}
{\Large Nº1\par}
\vspace{1.2cm}

{\large Alumnos: Cesana Andrés, Pieckenstainer Mateo, Ricci Matias, Trucchi Genaro.\par}
\vspace{0.4cm}
{\large Profesor: Ferreyra Pablo.\par}
\vfill
{\large Año 2024\par}
\end{titlepage}

\tableofcontents
\newpage

% =====================================================
\section{Introducción}

Este es un trabajo realizado en la materia Síntesis de Redes Activas perteneciente al
octavo cuatrimestre de la carrera de ingeniería electrónica.
Todo el marco práctico de la experiencia se realizó en la Facultad de Ciencias Exactas
Físicas y Naturales de la Universidad Nacional de Córdoba.
El objetivo de este proyecto es familiarizarse con el armado y análisis de circuitos
analógicos lineales y no lineales con respecto a los amplificadores operacionales ideales.

% =====================================================
\section{Marco Teórico}

Los amplificadores lineales integrados son circuitos electrónicos diseñados para
amplificar señales analógicas de manera proporcional, manteniendo la relación lineal entre la
entrada y la salida dentro de su rango de operación. Estos dispositivos suelen incluir varias
etapas internas, como amplificación diferencial, ganancia en tensión y salida. Poseen dos
entradas, una conocida como no inversora (V+) y otra denominada inversora (V-), y una
salida (Vo). La salida Vo es el resultado de la diferencia de potencial entre las entradas
multiplicadas por la ganancia a lazo abierto del amplificador normalmente conocida como Ad
(Vo=Ad (V+-V-).

En nuestro caso, tomaremos al amplificador operacional como ideal, sin tener en
cuenta el efecto de los errores sobre la salida, y además que, el ancho de banda es infinito
(BW $\to \infty$), la impedancia de entrada del operacional infinita (Zi $\to \infty$), la ganancia en modo
diferencial infinita (Ad $\to \infty$), la ganancia en modo común nula (Ac $\to 0$) y la impedancia de
salida nula (Zo $\to 0$).

% =====================================================
\section{Desarrollo}

% -----------------------------------------------------
\subsection*{Circuito 1: Amplificador diferencial}

\begin{figure}[H]
\centering
\includegraphics[width=0.85\textwidth]{figura1.png}
\caption{Figura 1: Circuito propuesto 1}
\end{figure}

El valor de R = 10K$\Omega$ (Decidido por el grupo de alumnos)

\textbf{Datos:}
\begin{itemize}
\item Amplificador Operacional LM324.
\item Vcc = 10V
\item Vss = $-10$V
\item R1 = R2 = R3 = R4 = R5 = R
\end{itemize}

\textbf{Parámetros a analizar:}

\textbf{ANALÍTICO:}
\[
\left( VC = \frac{(V1 + V2)}{2} \quad VD = (V2 - V1) \right)
\]
Vo1 = f(V1, V2); Vo1 = f(VD, VC)\\
Vo2 = f(V1, V2); Vo2 = f(VD, VC)\\
Impedancia vista por las fuentes de señal.

\textbf{MEDICIÓN - SIMULACIÓN:}\\
Gráfico Entrada/Salida: Vo1 = f(V1) y Vo1 = f(V2) \quad Vss < V1 , V2 < Vcc\\
Gráfico Entrada/Salida: Vo1 = f(VC) y Vo2 = f(VC) \quad Vss < VC < Vcc

\subsubsection*{Cálculos:}

\[
V_{o1} = f(V_1) + f(V_2)
\]
\[
V_1 = V_c - \frac{V_d}{2}
\qquad
V_2 = V_c + \frac{V_d}{2}
\]
\[
V_c = \frac{(V_1 + V_2)}{2}
\qquad
V_d = V_2 - V_1
\]

\[
V_{o1}\Big|_{(v1=0)} = -\frac{R2}{R1}\,V_2 = -V_2
\quad (\text{Configuración inversora})
\]

\[
V_{o1}\Big|_{(v2=0)} =
\left(\frac{R2}{R1//R4}+1\right)V_1 = 3V_1
\quad (\text{Configuración no inversora})
\]

Tenemos entonces por superposición que:
\[
V_{o1} = V_{o1}\Big|_{(v1=0)} + V_{o1}\Big|_{(v2=0)}
\]
\[
V_{o1} = 3V_1 - V_2
\]

Realizamos ahora lo mismo para Vo2

\[
V_{o2}\Big|_{(v1=0 \; y \; v2=0)} = -\frac{R3}{R5}V_{o1}
= -V_{o1} = -3V_1 + V_2
\]

\[
V_{o2}\Big|_{(vo1=0 \; y \; v2=0)} = -\frac{R3}{R1}V_1 = -V_1
\]

\[
V_{o2}\Big|_{(vo1=0 \; y \; v1=0)} =
\left(\frac{R3}{R1//R5}+1\right)V_2 = 3V_2
\]

Tenemos entonces por superposición que:
\[
V_{o2} =
V_{o2}\Big|_{(v1=0 \; y \; v2=0)}
+
V_{o2}\Big|_{(vo1=0 \; y \; v2=0)}
+
V_{o2}\Big|_{(vo1=0 \; y \; v1=0)}
\]

\[
V_{o2} = 4(V_2 - V_1)
\]
\[
V_{o2} = 4(V_d + 0V_c)
\]
\[
V_o = 4(V_2 - V_1) = 4(V_d + 0V_c)
\]

En la configuración de este circuito, notamos que la tensión de salida es la diferencia
entre las tensiones de entradas, y si aplicamos una tensión común como V1=Vc y V2=Vc con
Vd=0 tendremos que la salida para modo común será nula. Por ende, este circuito es capaz de
anular el modo común por completo. De lo analizado anteriormente obtenemos que Ad=4 y
Ac=0 por lo que RRMC =Ad/Ac $\to \infty$

\subsubsection*{Simulación:}

Realizamos la simulación del circuito en el simulador LT Spice.
\begin{itemize}
\item $V_{o1}$ y $V_{o2}$ con: $V1 = 0.5V$ y $V2 = 1V$
\item $V_{o1}$ y $V_{o2}$ con: $V1 = 1V$ y $V2 = 1V$
\end{itemize}

\begin{figure}[H]
\centering
\includegraphics[width=0.85\textwidth]{figura2.png}
\caption{Figura 2: Circuito simulado}
\end{figure}

\begin{figure}[H]
\centering
\includegraphics[width=0.85\textwidth]{figura3.png}
\caption{Figura 3: Vo1 (azul) y Vo2 (verde) con V1= 0.5 V y V2=1V}
\end{figure}

\begin{figure}[H]
\centering
\includegraphics[width=0.85\textwidth]{figura4.png}
\caption{Figura 4: Vo1 (azul) y Vo2 (verde) con V1= 1 V y V2=1V}
\end{figure}

% -----------------------------------------------------
\subsection*{Circuito 2: Fuente de corriente controlada por tensión}

\begin{figure}[H]
\centering
\includegraphics[width=0.75\textwidth]{figura5.png}
\caption{Figura 5: Circuito propuesto 2}
\end{figure}

\textbf{Datos:}
\begin{itemize}
\item Amplificador Operacional LM324.
\item Vcc = 10V
\item Vss = $-10$V
\item R1 = 100$\Omega$; R2 = 10K$\Omega$; R3 = 1K$\Omega$; R4 = 100K$\Omega$
\end{itemize}

\textbf{Parámetros a analizar:}\\
IRL = f(RL, VIN); Vo = f(VIN, RL); RLmax = f(VIN)

\subsubsection*{Cálculos:}

Para analizar el circuito propuesto, se propone expresar V + (la entrada no inversora del AO)
y V − (la entrada inversora del AO) en funci´on del Vo, planteando el divisor resistivo en el
nodo ”2”de la figura:

\[
V^+ = V^- = V_o \frac{R4}{R4+R2}
\]

Y planteamos Kirchhoff en el nodo 3:

\[
\frac{V_{in}-V^+}{R3} + \frac{V_o - V^+}{R1} = \frac{V^+}{RL}
\]

\[
\frac{V_{in}}{R3} + \frac{V_o}{R1} = V^+\left(\frac{1}{RL}+\frac{1}{R1}+\frac{1}{R3}\right)
\]

Y reemplazamos $V^+$:

\[
\frac{V_{in}}{R3} + \frac{V_o}{R1}
=
V_o \frac{R4}{R4+R2}
\left(
\frac{1}{RL} + \frac{1}{R1} + \frac{1}{R3}
\right)
\]

\[
V_{in} =
V_o \left[
\frac{1}{RL}
\left(
R3 \frac{R4}{R4+R2}
\right)
+
\left(
R3 \frac{R4}{R4+R2}
\right)
\left(
\frac{1}{R1} + \frac{1}{R3}
\right)
-
\frac{R3}{R1}
\right]
\]


Reemplazamos los valores de las resistencias:

\[
V_{in} = V_o\left[\frac{1}{RL}(909,09091)\right]
\]

La corriente de la carga es:

\[
I_{RL}=\frac{V^+}{RL}
\]

\[
I_{RL} = V_o\frac{R4}{R4+R2}\frac{1}{RL}
\]

\[
I_{RL}=\frac{V_{in}}{R3}
\]

\[
I_{RL} = V_{in}\,10^{-3}
\]

Y definimos la tensión Vo en función de RL y de Vin:

\[
V_o=\frac{V_{in}}{\left(\frac{1}{RL}(909,09091)\right)}
\]

\[
V_o = V_{in}\,RL\,(1.1 \cdot 10^{-3})
\]

Ahora, se determina el valor de RLmax, que, como el AO es ideal, la tensión de salida
máxima será VCC=10 V. Por lo que se obtiene:

\[
RL_{max}=\frac{9090}{V_{in}}
\]

Se pueden realizar las siguientes tablas:

\begin{table}[H]
\centering
\caption{Tabla 1: Valores teóricos de IRL en funcion de RL y de Vin}
\begin{tabular}{c c c c}
\toprule
\textbf{IRL[$\mu$A]} & \textbf{0.5} & \textbf{1} & \textbf{2} \\
\midrule
5*RL[K$\Omega$] & 0 & 0 & 0 \\
1 & 500 & 1000 & 2000 \\
2 & 500 & 1000 & 2000 \\
5 & 500 & 1000 & 2000 \\
10 & 500 & 1000 & 2000 \\
\bottomrule
\end{tabular}
\end{table}

\begin{table}[H]
\centering
\caption{Tabla 2: Valores teóricos de Vo en función de RL y de Vin}
\begin{tabular}{c c c c}
\toprule
\textbf{Vo[V]} & \textbf{0.5} & \textbf{1} & \textbf{2} \\
\midrule
5*RL[K$\Omega$] & 0 & 0 & 0 \\
1 & 0.55 & 1.1 & 2.2 \\
2 & 1.1 & 2.2 & 4.4 \\
5 & 2.75 & 5.5 & 11 \\
10 & 5.5 & 11 & 22 \\
\bottomrule
\end{tabular}
\end{table}

Los valores que superen la tensión Vcc, la salida se enclava en ese mismo valor y la onda se
recortará

\subsubsection*{Simulación:}

\begin{figure}[H]
\centering
\includegraphics[width=0.85\textwidth]{figura6.png}
\caption{Figura 6: Circuito simulado}
\end{figure}

\begin{figure}[H]
\centering
\includegraphics[width=0.85\textwidth]{figura7.png}
\caption{Figura 7: Corriente de la carga en función de RL y Vin}
\end{figure}

Se realizó un barrido en continua de 0 a 10 V, y se puede ver que la variación de corriente es
lineal hasta el punto de tensión de cada valor de la resistencia de carga que satura al AO

\begin{table}[H]
\centering
\begin{tabular}{c c}
\toprule
RL & Vin-sat \\
\midrule
1K & 7,4V \\
2K & 3,9V \\
5K & 1,5V \\
10K & 0,8V \\
\bottomrule
\end{tabular}
\end{table}

\begin{figure}[H]
\centering
\includegraphics[width=0.85\textwidth]{figura8.png}
\caption{Figura 8: Vo en función de RL y Vin}
\end{figure}

Vemos que la tensión de entrada y la resistencia RL, varían la salida Vo

\begin{table}[H]
\centering
\caption{Cuadro 3: Valores simulados de IRL en función de RL y Vin}
\begin{tabular}{c c c c}
\toprule
\textbf{IRL[$\mu$A]} & \textbf{0.5} & \textbf{1} & \textbf{2} \\
\midrule
RL[K$\Omega$] & 0 & 0 & 0 \\
1 & 495 & 994 & 1992 \\
2 & 494 & 992 & 1989 \\
5 & 493 & 990 & 1567 \\
10 & 489 & 777 & 780 \\
\bottomrule
\end{tabular}
\end{table}

% -----------------------------------------------------
\subsection*{Circuito 3: Rectificador de precisión}

\begin{figure}[H]
\centering
\includegraphics[width=0.80\textwidth]{figura9.png}
\caption{Figura 9: Circuito propuesto 3}
\end{figure}

Amplificador Operacional LM324.\\
Vcc = 10V\\
Vss = $-10$V\\
R1 = 100$\Omega$; R2 = 10K$\Omega$; R3 = 1K$\Omega$; R4 = 100K$\Omega$

\textbf{Parámetros a analizar:}\\
Vo1 = f(Vin), Vo2 = f(Vin); con 0V < Vin (Ignorar Rd del diodo)\\
Vo1 = f(Vin), Vo2 = f(Vin); con Vin > 0V (Ignorar Rd del diodo)

\subsubsection*{Cálculos:}

Debemos realizar el análisis para dos condiciones de funcionamiento distintas:
\begin{itemize}
\item Vin>0 entonces el diodo 1 no conducirá, y el diodo 2 si lo hará. La resistencia R1,
producirá realimentación debido a que entre R1 y R2 no habrá diferencia de potencial.
Entendemos entonces que el amplificador 1, está trabajando como seguidor de
tensión.
\item Vin<0 entonces el diodo 1 no conducira, y el diodo 2 entrará en conducción
\end{itemize}

Analizamos para la primera condición:

\[
V_o\Big|_{v^+=V_i}=\left(\frac{R4}{R2+R4}+1\right)V_{in}=\frac{5}{3}V_{in}
\]

\[
V_o\Big|_{v^-=V_i}=-\frac{R1}{R2+R1}V_{in}=-\frac{2}{3}V_{in}
\]

Entonces $\rightarrow V_{oT}=V_{in}$

Analizamos para la segunda condición:

\[
V_o\Big|_{v^+=V_i}=\left(\frac{R4}{R4}+1\right)V_{in}=3V_{in}
\]

\[
V_o\Big|_{v^-=V_i}=-\frac{R4}{R1}V_{in}=-4V_{in}
\]

Entonces $\rightarrow V_{oT}=-V_{in}$

\subsubsection*{Simulación:}

\begin{figure}[H]
\centering
\includegraphics[width=0.85\textwidth]{figura10.png}
\caption{Figura 10: Circuito simulado}
\end{figure}

Utilizamos una señal senoidal de 2 V de amplitud, para ver el comportamiento de la tensión
de salida de ambos amplificadores.

\begin{figure}[H]
\centering
\includegraphics[width=0.85\textwidth]{figura11.png}
\caption{Figura 11: Vo1 (azul) y Vo2 (rojo) con Vin=2V (verde)}
\end{figure}

\begin{figure}[H]
\centering
\includegraphics[width=0.85\textwidth]{figura12.png}
\caption{Figura 12: Vo1 (azul) y Vo2 (rojo) con Vin=4.5V (verde). Este es el límite de saturación de la salida del primer amplificador}
\end{figure}

\begin{figure}[H]
\centering
\includegraphics[width=0.85\textwidth]{figura13.png}
\caption{Figura 13: Vo1 (azul) y Vo2 (rojo) con Vin=5V (verde). La rectificación de la señal se ve distorsionada por la saturación del primer OA.}
\end{figure}

Se puede ver entonces que el rectificador funciona bien dentro del rango -4,5 V y 4,5 V. Sin
embargo, para valores que no se encuentran en dicho rango, la salida se distorsiona debido a
la saturación de la salida del primer OA

% -----------------------------------------------------
\subsection*{Circuito 4: Comparador con histéresis}

\begin{figure}[H]
\centering
\includegraphics[width=0.70\textwidth]{figura14.png}
\caption{Figura 14: Circuito propuesto 4}
\end{figure}

Amplificador Operacional LM324.\\
V+ = 10V\\
V- = 0V\\
R1 = R2 = R4 = 10K$\Omega$; R3 = 2K$\Omega$\\
Vref=2V

\textbf{Parámetros a analizar:}
\begin{itemize}
\item Umbral de conmutación cuando Vo = V+
\item Umbral de conmutacion cuando Vo = V-
\item Vo = f(Vin) con V- < Vin < V+
\end{itemize}

\subsubsection*{Cálculos:}

Consideramos el caso general con alimentación simétrica. Se analiza también Vo a partir de
la tensión diferencial y el caso particular con Vss=0. Se definen v+ y v- así:

\[
v^- = k_1 \cdot V_{in}
\qquad
v^+ = k_2 (V_o - V_{ref}) + V_{ref}
\]

Teniendo que:

\[
k_1=\frac{R2}{R1+R2}
\qquad
k_2=\frac{R3}{R3+R4}
\]

Si Vd<0, la salida del amplificador debería ser Vss:

\[
V_d = (v^+ - v^-) < 0 \Rightarrow V_o = V_{ss}
\]

Entonces:

\[
v^+ < v^- \Rightarrow
k_2 (v_o - v_{ref}) + v_{ref} < k_1 v_{in}
\]

\[
\frac{k_2}{k_1}(v_o - v_{ref}) + \frac{v_{ref}}{k_1} < v_{in}
\]

\[
\frac{k_2}{k_1}v_o + \frac{1-k_2}{k_1}v_{ref} < v_{in}
\]

\[
v_o = v_{cc}
\qquad
v_{cc}=10V
\qquad
v_{ref}=2V
\]


\[
v_{in} > 6,67V \Rightarrow v_o = v_{ss}
\]

Ahora si Vd>0, la salida de un AO ideal debería ser Vcc:

\[
V_d = (v^+ - v^-) > 0 \Rightarrow V_o = V_{cc}
\]

Entonces:

\[
v^+ > v^- \Rightarrow
k_2 (v_o - v_{ref}) + v_{ref} > k_1 v_{in}
\]

\[
\frac{k_2}{k_1}(v_o - v_{ref}) + \frac{v_{ref}}{k_1} > v_{in}
\]

\[
\frac{k_2}{k_1}v_o + \frac{1-k_2}{k_1}v_{ref} > v_{in}
\]

\[
v_o = v_{ss}
\qquad
v_{cc}=-10V
\qquad
v_{ref}=2V
\]

R3 = 2K$\Omega$\\
R1 = R2 = R4 = 10K$\Omega$

\[
v_{in} < 0V \Rightarrow v_o = v_{cc}
\]

Entonces tenemos que:

\[
v_o(v_{in}) = v_{cc} \;\; \text{para} \;\; v_{in} < 0V
\]
\[
v_o(v_{in}) = v_{ss} \;\; \text{para} \;\; v_{in} > 6,67V
\]

Si Vss=0, entonces la relación queda:

\[
\frac{1-k_2}{k_1} v_{ref} > v_{in}
\]

Donde se ve que el punto de conmutación queda directamente dependiente de vref

Reemplazamos por los valores. entonces queda:

Si: $v_{in} < 3,33V \Rightarrow v_o = v_{ss}$

Los análisis anteriores son válidos para un amplificador ideal o de tipo Rail to Rail, en este
caso obtenido por simulación sa salida máxima del AO es de 8,5 V, entonces queda:

Si: $v_{in} > 6,17V \Rightarrow v_o = v_{cc}$

\subsubsection*{Simulación:}

\begin{figure}[H]
\centering
\includegraphics[width=0.85\textwidth]{figura15.png}
\caption{Figura 15: Circuito simulado}
\end{figure}

\begin{figure}[H]
\centering
\includegraphics[width=0.85\textwidth]{figura16.png}
\caption{Figura 16: Vo1=f(vin) con Vin= 5V y 1KHz}
\end{figure}

\begin{figure}[H]
\centering
\includegraphics[width=0.85\textwidth]{figura17.png}
\caption{Figura 16: Vo1=f(vin) con Vin= 5V y 1KHz}
\end{figure}

Se puede ver así, como a diferencia de un comparador normal que compara dos voltajes de
entrada y genera una señal de salida alta o baja según cuál de ellos sea mayor, un comparador
con histéresis incorpora una especie de memoria en el circuito, lo cual implica que, una vez
que la salida ha cambiado de estado,se necesita un cambio más significativo en la senal de
entrada para que la salida vuelva a conmutar. Esta característica es especialmente útil en
aplicaciones como la detección de niveles, el control de motores y los circuitos de disparo, ya
que ayuda a evitar cambios erráticos en la salida debido a ruidos o fluctuaciones pequeñas en la
señal de entrada.

% =====================================================
\section{Conclusión}

En el trabajo, se pudo relacionar conceptos con respecto a los amplificadores operaciones, y
sus usos básicos, combinados con componentes simples con diferentes disposiciones en los
circuitos.
Ya sea con el amplificador diferencial, la fuente de corriente controlada por tensión, el
rectificador de precisión o el comparador por histéresis, fue interesante el poder jugar y variar
los valores, para observar los diferentes comportamientos de los circuitos, como así también,
tener una nocion y practica, por sobre estos mismos.
Creemos que el trabajo fue eficiente y fructífero para nuestro conocimiento y estamos
satisfechos con el trabajo realizado.

\end{document}
