\section{Oscilador de anillo}
Un oscilador se define como un circuito capaz de producir una salida periódica autónoma, prescindiendo de una señal de excitación continua externa. Dependiendo de la aplicación, estas señales pueden adoptar formas sinusoidales, cuadradas o triangulares. Debido a su versatilidad, son componentes críticos en la infraestructura de telecomunicaciones, la temporización de relojes y la lógica digital. Se los puede clasificar como:
\begin{itemize}
    \item \textbf{Sinusoidales:} Se caracterizan por transiciones suaves y continuas. Ejemplos comunes incluyen las topologías LC y RC.
    \item \textbf{No Sinusoidales:} Generan ondas con flancos abruptos o cambios repentinos, como es el caso de los multivibradores o los osciladores de anillo.

\end{itemize}

El desafío principal en el diseño de estos dispositivos reside en asegurar la estabilidad de frecuencia y la pureza de la señal, factores que están intrínsecamente ligados a la calidad de los componentes y a las condiciones ambientales de operación.