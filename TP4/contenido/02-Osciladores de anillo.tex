\subsection{Introducción a los osciladoreess de anillo}
Un oscilador de anillo es un circuito electrónico compuesto por un número impar de inversores (puertas NOT) conectados en cascada y en lazo cerrado. Esta configuración inestable genera una señal cuadrada que alterna continuamente entre niveles lógicos altos y bajos, utilizada frecuentemente como reloj en circuitos digitales.
La frecuencia de oscilación está determinada por el tiempo de propagación de las etapas y se puede expresar como:

\[
f = \frac{1}{2n \cdot t_p}
\]

donde $n$ es el número de etapas y $t_p$ es el tiempo de propagación de cada inversor.
El núcleo de un oscilador de anillo reside en una cadena de inversores CMOS dispuestos en serie, donde se establece un lazo de retroalimentación conectando la salida de la etapa final directamente a la entrada de la primera. La operación de este circuito depende intrínsecamente de los retardos de propagación causados por las capacitancias parásitas de los transistores, las cuales son esenciales para sostener la oscilación.

\begin{figure} [H]
    \centering
    \includegraphics[width=0.5\linewidth]{Imagenes/Anillo.png}
    \caption{Configuracíon del oscilador de anillo}
    \label{fig:anillo}
\end{figure}